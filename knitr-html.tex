% Options for packages loaded elsewhere
\PassOptionsToPackage{unicode}{hyperref}
\PassOptionsToPackage{hyphens}{url}
%
\documentclass[
]{article}
\title{Case Study 01}
\author{Alexy Morris}
\date{10/11/2021}

\usepackage{amsmath,amssymb}
\usepackage{lmodern}
\usepackage{iftex}
\ifPDFTeX
  \usepackage[T1]{fontenc}
  \usepackage[utf8]{inputenc}
  \usepackage{textcomp} % provide euro and other symbols
\else % if luatex or xetex
  \usepackage{unicode-math}
  \defaultfontfeatures{Scale=MatchLowercase}
  \defaultfontfeatures[\rmfamily]{Ligatures=TeX,Scale=1}
\fi
% Use upquote if available, for straight quotes in verbatim environments
\IfFileExists{upquote.sty}{\usepackage{upquote}}{}
\IfFileExists{microtype.sty}{% use microtype if available
  \usepackage[]{microtype}
  \UseMicrotypeSet[protrusion]{basicmath} % disable protrusion for tt fonts
}{}
\makeatletter
\@ifundefined{KOMAClassName}{% if non-KOMA class
  \IfFileExists{parskip.sty}{%
    \usepackage{parskip}
  }{% else
    \setlength{\parindent}{0pt}
    \setlength{\parskip}{6pt plus 2pt minus 1pt}}
}{% if KOMA class
  \KOMAoptions{parskip=half}}
\makeatother
\usepackage{xcolor}
\IfFileExists{xurl.sty}{\usepackage{xurl}}{} % add URL line breaks if available
\IfFileExists{bookmark.sty}{\usepackage{bookmark}}{\usepackage{hyperref}}
\hypersetup{
  pdftitle={Case Study 01},
  pdfauthor={Alexy Morris},
  hidelinks,
  pdfcreator={LaTeX via pandoc}}
\urlstyle{same} % disable monospaced font for URLs
\usepackage[margin=1in]{geometry}
\usepackage{graphicx}
\makeatletter
\def\maxwidth{\ifdim\Gin@nat@width>\linewidth\linewidth\else\Gin@nat@width\fi}
\def\maxheight{\ifdim\Gin@nat@height>\textheight\textheight\else\Gin@nat@height\fi}
\makeatother
% Scale images if necessary, so that they will not overflow the page
% margins by default, and it is still possible to overwrite the defaults
% using explicit options in \includegraphics[width, height, ...]{}
\setkeys{Gin}{width=\maxwidth,height=\maxheight,keepaspectratio}
% Set default figure placement to htbp
\makeatletter
\def\fps@figure{htbp}
\makeatother
\setlength{\emergencystretch}{3em} % prevent overfull lines
\providecommand{\tightlist}{%
  \setlength{\itemsep}{0pt}\setlength{\parskip}{0pt}}
\setcounter{secnumdepth}{-\maxdimen} % remove section numbering
\ifLuaTeX
  \usepackage{selnolig}  % disable illegal ligatures
\fi

\begin{document}
\maketitle

\#With the Beer and Breweries dataset provided by you, the CEO and CFO
of Budweiser, our team was able to answer the questions you presented
and have even addressesed some additional questions that arose. As a
disclaimer we are working on a relatively local scale so all information
should only be eneralized the the United States of America, our team
would not recommend utalizing the same infomation obroad. Overall our
team firmly believes that the following work will reflect the current
and future trends in the craft beer market.

\#1-How many breweries are present in each state?

\begin{verbatim}
map <- Breweries %>% count(State)
map

map$State <- state.name[match(map$State,state.abb)]
map 

names(map)[names(map) == "State"] <- "state"

library(usmap)
library(ggplot2)

plot_usmap(data = map, values = "n", color = "red") + 
  scale_fill_continuous(name = "Brewery Number", label = scales::comma) + 
  theme(legend.position = "right")
\end{verbatim}

\#In this section, we automated the counting process using the 51 states
as catcategories and then counted the number of times said category
appears in our data.

\#Through the 50 states we see that the highest number of breweries is
in Colorado with 47 and one of the lowest is in South Dakota with 1.
That said most states are in the middle with between about 10 to 30.

\#2-Merge beer data with the breweries data. Print the first 6
observations and the last six observations to check the merged file.

\begin{verbatim}
library(dplyr)

x <- data.frame(Beers)
y <- data.frame(Breweries)
names(x)[names(x) == "Brewery_id"] <- "Brew_ID"
names(x)[names(x) == "Name"] <- "Beer_Name"

Case <- merge(x, y, by="Brew_ID", all = TRUE)
View(Case)

head(Case, n=6)

tail(Case, n=6)
\end{verbatim}

\#For convienience we combined the 2 dataframes given by matching up the
brewery id numbers. We also took the time to rename a few columns for
easy identification.

\#3-Address the missing values in each column

\begin{verbatim}
colSums(is.na(Case))
\end{verbatim}

\#In this sections code we simply told our program to count/sum every
missing value in each column.

\#While giving the data a cursurary glance our team noticed a number of
missing values specifically in the IBU and ABV columns; unfortounely
upon closer examination we see there are 1000+ missing values across our
columns. The volume of missing vales means that it would be incredibly
labour intensive on our teams part to fill the blanks manuelly.

\#4-Compute the median alcohol content and international bitterness unit
for each state. Plot a bar chart to compare.

\begin{verbatim}
alcohol = Case %>% select(Name, ABV, State) %>% filter(!is.na(ABV))
bitterness = Case %>% select(Name, IBU, State) %>% filter(!is.na(IBU))

MM_Alc <- alcohol %>% group_by(State) %>% summarise(Max=max(ABV), Min=min(ABV), Median=median(ABV))

MM_Alc_tophalf <- head(MM_Alc, n=25)
MM_Alc_bothalf <- tail(MM_Alc, n=26)

barplot(height=MM_Alc_tophalf$Median, names=MM_Alc_tophalf$State, col="#69b3a2", horiz=T, las=1)
barplot(height=MM_Alc_bothalf$Median, names=MM_Alc_bothalf$State, col="#69b3a2", horiz=T, las=1)
  
MM_bit <- bitterness %>% group_by(State) %>% summarise(Ma=max(IBU), Mi=min(IBU), Med=median(IBU))

MM_bit_tophalf <- head(MM_bit, n=25)
MM_bit_bothalf <- tail(MM_bit, n=26)

barplot(height=MM_bit$Med, names=MM_bit$State, col="#69b3d2",horiz=T, las=1)

barplot(height=MM_bit_tophalf$Med, names=MM_bit_tophalf$State, col="#69b3d2", horiz=T, las=1)
barplot(height=MM_bit_bothalf$Med, names=MM_bit_bothalf$State, col="#69b3d2", horiz=T, las=1)
\end{verbatim}

\#To compute the median ABV and IBU level for each state we first broke
down the larger given dataframe to only the needed variables in order to
save on computation times. Then we got to the relevant calculations and
lastly broke the medians into two dataframes

\#Our bar graphs revealed that, in regards to median, ABV is mostly
consistent across statelines but on the other hand there was a lot of
variety when it came to IBU.

\#5-Which state has the maximum alcoholic (ABV) beer? Which state has
the most bitter (IBU) beer?

\begin{verbatim}
M_ABV = Case %>% slice_max(ABV, n = 5)
M_ABV

M_IBU = Case %>% slice_max(IBU, n = 5)
M_IBU
\end{verbatim}

\#For the fifth question we simply reorginized the graphs from greatest
ABV to least ABV and again from greatest to least for IBU.

\#The state with the highest ABV is Colorado and the state wth the
highest IBU is Oregan.

\#6-Comment on the summary statistics and distribution of the ABV
variable.

\begin{verbatim}
summary(Case$ABV)
hist(Case$ABV, main="ABV Distribution", xlab="ABV", col="red")
\end{verbatim}

\#By using the baseline functions of R we get a range of standard
statistical values and even a histogram based off of our data.

\#Both of graph and our summary statistics paint a pincture of a normal
distribution. Our mean and median are very close together and our 1st
and 3rd quartiles appear to be inline with the expectations of a normal
distribution.

\#7- Is there an apparent relationship between the bitterness of the
beer and its alcoholic content? Draw a scatter plot.

\begin{verbatim}
ggplot(data = Case, mapping = aes(x = ABV, y = IBU))+ geom_point() + geom_smooth(method = lm) 

lm(ABV ~ IBU, data=Case)
\end{verbatim}

\#To address the relationship between ABV and IBU we have graphed IBU
with respecet to ABV and created a linear model to highlight the direct
relationship between the two.

\hypertarget{we-notice-that-there-does-appear-to-be-a-direct-positive-relationship-between-abv-and-ibu.}{%
\section{We notice that there does appear to be a direct positive
relationship between ABV and
IBU.}\label{we-notice-that-there-does-appear-to-be-a-direct-positive-relationship-between-abv-and-ibu.}}

\#8-Budweiser would also like to investigate the difference with respect
to IBU and ABV between IPAs\ldots{}

\begin{verbatim}
library(tidyverse)
library(class)
library(caret)

Case$TIPA <- grepl("IPA", Case$Style, ignore.case = T)
Case$TAle <- grepl("Ale", Case$Style, ignore.case = T)

Base_IPA = Case %>% select(ABV, IBU, TIPA) %>% filter(!is.na(ABV)) %>% filter(!is.na(IBU))

test = data.frame(ABV = .05, IBU = 75)
 
knn(Base_IPA[,c(1,2)], test, Base_IPA$TIPA, k = 5, prob = TRUE)
knn(Base_IPA[,c(1,2)], test, Base_IPA$TIPA, k = 15, prob = TRUE)
\end{verbatim}

\#For our KNN classification we have taken the essentially the same
scatter plot from above and plotted different test points to examine the
likely of a beer with our same ABV and IBU being a IPA or an Ale.

\#Form a number of test we see that with our given test we were much
more likely to get IPA as True in the case of a high IBU with the ABV
seemingly not mattering.

\#In conclusion, through our research into the beer/brewery dataframes
we have uncovered many answer to questions both requested by you, our
client, and observed from notable trends. We hope these results provided
the desired conclusions and in the event your company wants for a deeper
analyzation we hope you will keep our team in mind. ```

\end{document}
